\documentclass[12pt,openany]{report}
\usepackage[a4paper,left=1in,right=1in,top=1in,bottom=1in,footskip=.25in]{geometry}
\usepackage[utf8x]{inputenc}
\usepackage[english]{babel}
\setlength{\parskip}{1em}
\usepackage{url}
\usepackage{palatino}
\usepackage{tocloft}
\usepackage{marginnote}
\usepackage{amssymb,amsmath,amsthm,amsfonts}
\usepackage{mathtools}
\usepackage{hyperref}
\hypersetup{
    colorlinks,
    citecolor=black,
    filecolor=black,
    linkcolor=black,
    urlcolor=blue
}
\newtheorem{theorem}{Theorem}
\renewcommand\cftchapafterpnum{\vskip5pt}
\renewcommand\cftsecafterpnum{\vskip0pt}

\usepackage{minitoc}
\begin{document}

\begin{titlepage}
    \begin{center}
        \vspace*{1cm}
        
        \Huge
        \textbf{Empirical Finance: A Review}

        \Large
        \textit{For Personal Reference}
            
        \vspace{2.5cm}
        
        \LARGE    
        \textbf{Sai Zhang}
            
        \vfill
        
        \large    
        Inspired by the Course \textbf{\textit{Empirical Finance}} at\\
        London Business School by \href{https://sabryzgalova.com/research/}{\textit{Dr. Svetlana Bryzgalova}}
            
        \vspace{0.8cm}
        \large
        \today
            
    \end{center}
\end{titlepage}

%%%%%%%% Main content %%%%%%%%%%
%%%%%%%% Chapters are called separately %%%%%%%%%%

\chapter*{Here we go!}
Empirical finance is an absolutely fascinating field, with some of 
the most cutting-edge methodologies and the most exploratory 
techniques. Although it is not my speciality, I am always interested
in this literature. During my pre-doc research fellowship at London
Business School, I have had the previlege to study in the course
\textit{\textbf{Financial Economics II: Empirical Finance}}. The course
instructor 

\newpage

\dominitoc
\tableofcontents

\chapter{This is Chapter 1}
\minitoc

\vspace{0.5cm}
Every investor knows that trading in financial markets is to play
games with time itself. Daily trades determine asset prices at every date and hence
influence the random distribution of future prices as well as the initial
level of prices. One would need "much more careful attention to the process
by which both expected payoffs and required rates of return determine
asset prices"\footnote{See \citet[p.~121]{campbell2017financial}}.

In this chapter, I first, following \citet[Chapter~5]{campbell2017financial}, summarize models 
mapping cash flows and discount rates into prices using present value relations in Section \ref{chap1:sec1}.
Then I discuss the early evidence for mean reversion in returns in Section \ref{chap1:sec2}.
In Section \ref{chap1:sec3}, I examine the excess volatility puzzle in the predictability debate.
To accomodate the stylized facts of time-series predictability, Section \ref{chap1:sec4}
presents two of the most influential approaches to decompose prices. In Section \ref{chap1:sec5},
I discuss the so-called "Prediction Zoo"

\section{Mean Reversion in Returns: Early Evidence}\label{chap1:sec1}
Section 1: aaaaaaaaa

\section{Mean reversion in returns: early evidence}\label{chap1:sec2}
Section 2: bbbbbbbbb

\section{Excess volatility puzzle}\label{chap1:sec3}
Section 3: cccccccc

\section{Decomposing prices}\label{chap1:sec4}
\subsection{Campbell-Schiller Decomposition}\label{chap1:sec4:ssec1}
\subsection{}

\section{Prediction zoo}\label{chap1:sec5}

\chapter{This is Chapter 2}
\minitoc

\vspace{0.5cm}

Intro:

\section{Section 1}
Section 1:

\end{document}