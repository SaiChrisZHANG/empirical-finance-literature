\documentclass[12pt,openany]{report}
\usepackage[a4paper,left=1in,right=1in,top=1in,bottom=1in,footskip=.25in]{geometry}
\usepackage[utf8x]{inputenc}
\usepackage[english]{babel}
\setlength{\parskip}{1em}
\usepackage{url}
\usepackage{palatino}
\usepackage{tocloft}
\usepackage{marginnote}
\usepackage{natbib}
\usepackage{amssymb,amsmath,amsthm,amsfonts}
\usepackage{mathtools}
\usepackage{hyperref}
\hypersetup{
    colorlinks,
    citecolor=black,
    filecolor=black,
    linkcolor=black,
    urlcolor=blue
}
\newtheorem{theorem}{Theorem}
\renewcommand\cftchapafterpnum{\vskip5pt}
\renewcommand\cftsecafterpnum{\vskip0pt}

\usepackage{minitoc}

\begin{document}

\begin{titlepage}
    \begin{center}
        \vspace*{1cm}
        
        \Huge
        \textbf{Empirical Finance: A Review}

        \Large
        \textit{For Personal Reference}
            
        \vspace{2.5cm}
        
        \LARGE    
        \textbf{Sai Zhang}
            
        \vfill
        
        \large    
        Inspired by the Course \textbf{\textit{Empirical Finance}} at\\
        London Business School by \href{https://sabryzgalova.com/research/}{\textit{Dr. Svetlana Bryzgalova}}
            
        \vspace{0.8cm}
        \large
        \today
            
    \end{center}
\end{titlepage}

%%%%%%%% Main content %%%%%%%%%%
%%%%%%%% Chapters are called separately %%%%%%%%%%

\chapter*{Here we go!}

Empirical finance is an absolutely fascinating field, with some of 
the most cutting-edge methodologies and the most exploratory 
techniques. Although it is not my speciality, I am always interested
in this literature. During my pre-doc research fellowship at London
Business School, I have had the previlege to study in the course
\textit{\textbf{Financial Economics II: Empirical Finance}}. The course
instructor Dr. Svetlana Bryzgalova is absolutely one of the most brilliant
scholars I have encountered. Thanks to her, I have got to understand 
this liturature more systematically. In this (personal) review, I summarize
the most influential and inspirational works in this field and organize
them by different topics. The structure of this review resembles the structure
of Dr. Bryzgalova's course, but adjusted according to my personal research
interest. I intend to review classic works and discuss some potential directions
of future study regarding my personal interest in Behavioral Economics, Game Theory
and Network.

Since this review is tailored according to my own research interest and
experience, I will not only summarize the theoretical perspectives of the
studies, present their findings and discuss how they fit into the literature,
but document my replication attempts and pseudo codes as well. All the codes related
to this review can be found on \href{https://github.com/SaiChrisZHANG}{my Github page}.

I thank Dr. Svetlana Bryzgalova for her valuable intuitions and impressive
knowledge of the empirical finance literature. Building this review is truly
a memorable journey for me. I would love to share this review and all the related
materials to anyone that finds them useful. And unavoidably, I would make some
typos and other minor mistakes (hopefully not big ones). So I'd really appreciate
any correction. If you find any mistakes, please either set up a branch on Github
or send the mistakes to this email address 
\href{mailto:saizhang.econ@gmail.com}{saizhang.econ@gmail.com}, BIG thanks in advance!

\newpage

\dominitoc
\tableofcontents

\chapter{Time-Series Predictability}
\minitoc

\vspace{0.5cm}
Every investor knows that trading in financial markets is to play
games with time itself. Daily trades determine asset prices at every date and hence
influence the random distribution of future prices as well as the initial
level of prices. One would need "much more careful attention to the process
by which both expected payoffs and required rates of return determine
asset prices"\footnote{See \citet[p.~121]{campbell2017financial}}.

In this chapter, I first, following \citet[Chapter~5]{campbell2017financial}, summarize models 
mapping cash flows and discount rates into prices using present value relations in Section \ref{chap1:sec1}.
Then I discuss the early evidence for mean reversion in returns in Section \ref{chap1:sec2}.
In Section \ref{chap1:sec3}, I examine the excess volatility puzzle in the predictability debate.
To accomodate the stylized facts of time-series predictability, Section \ref{chap1:sec4}
presents two of the most influential approaches to decompose prices. In Section \ref{chap1:sec5},
I discuss the so-called "Prediction Zoo"

\section{Mean Reversion in Returns: Early Evidence}\label{chap1:sec1}
Section 1: aaaaaaaaa

\section{Mean reversion in returns: early evidence}\label{chap1:sec2}
Section 2: bbbbbbbbb

\section{Excess volatility puzzle}\label{chap1:sec3}
Section 3: cccccccc

\section{Decomposing prices}\label{chap1:sec4}
\subsection{Campbell-Schiller Decomposition}\label{chap1:sec4:ssec1}
\subsection{}

\section{Prediction zoo}\label{chap1:sec5}

\chapter{Cross-Section Predictability}
\minitoc

\vspace{0.5cm}

Intro:

\section{Section 1}
Section 1:

\newpage
\bibliographystyle{aea}
\bibliography{refs}

\end{document}