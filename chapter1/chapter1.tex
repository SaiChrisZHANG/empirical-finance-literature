\minitoc

\vspace{0.5cm}
Every investor knows that trading in financial markets is to play
games with time itself. Daily trades determine asset prices at every date and hence
influence the random distribution of future prices as well as the initial
level of prices. One would need "much more careful attention to the process
by which both expected payoffs and required rates of return determine
asset prices"\footnote{See \citet[p.~121]{campbell2017financial}}.

In this chapter, I first, following \citet[Chapter~5]{campbell2017financial}, summarize models 
mapping cash flows and discount rates into prices using present value relations in Section \ref{chap1:sec1}.
Then I discuss the early evidence for mean reversion in returns in Section \ref{chap1:sec2}.
In Section \ref{chap1:sec3}, I examine the excess volatility puzzle in the predictability debate.
To accomodate the stylized facts of time-series predictability, Section \ref{chap1:sec4}
presents two of the most influential approaches to decompose prices. In Section \ref{chap1:sec5},
I selectively summarize some researches from the so-called "Prediction Zoo", which satirically 
describes the floods of price predictors. Finally, I discuss the issues and extensions of time-series
predictability in Section \ref{chap1:sec6}.

\section{Mean Reversion in Returns: Early Evidence}\label{chap1:sec1}
[insert text]

\section{Mean reversion in returns: early evidence}\label{chap1:sec2}
[insert text]

\section{Excess volatility puzzle}\label{chap1:sec3}
[insert text]

\section{Decomposing prices}\label{chap1:sec4}
[insert text]

\subsection{Campbell-Schiller decomposition}\label{chap1:sec4:ssec1}
[insert text]

\subsection{Lettau-Ludvigson decomposition}\label{chap1:sec4:ssec2}
[insert text]

\section{Prediction zoo}\label{chap1:sec5}
[insert text]

\section{Issues and extensions}\label{chap1:sec6}
[insert text]
\subsection{Persistency of most regressors}\label{chap1:sec6:ssec1}
[insert text]
\subsection{Aggregate predictors without ex-ante choice}\label{chap1:sec6:ssec2}
[insert text]
\subsection{Instability in the prediction relation}\label{chap1:sec6:ssec3}
[insert text]
\subsection{Measurement}\label{chap1:sec6:ssec4}
[insert text]
