In this section, I follow \citet[Chapter~5]{campbell2017financial} and discuss some 
of the conceptual building blocks for the strand of time-series empirical finance literature.

\subsection{Market Efficiency}
An intuitive way of explaining \textbf{\textit{market efficiency}} is that efficient markets
are competitive and allow no easy ways to make economic profit. A more useful and testable
definition was given by \citet[p.~127]{malkiel1989efficient}:

\begin{quote}
    The market is said to be efficient with respect to some information set $\phi$,
    if security prices would be unaffected by revealing that information to all participants.
\end{quote}

Some event studies that measure market responses to news announcements can be interpreted
as tests of market efficiency regarding the announced information, but in general, this 
definition is not easy to test. On the other hand, \cite{malkiel1989efficient} gives a more testable alternative:
\begin{quote}
    Efficiency with respect to an information set $\phi$ implies that it is impossible to
    make economic profits by trading on the basis of $\phi$.
\end{quote}

This is the idea behind an enormous literature in empirical asset pricing: if an economic model
defines the equilibrium return as $\Theta_{i,t}$, then the null hypothesis is
\begin{equation}
    R_{i,t+1} = \Theta_{i,t}+U_{i,t+1}
\end{equation}
where $U_{i,t+1}$ is a FAIR game regarding the information set at t, or $\mathbb{E}(U_{i,t+1}|\phi_t)=0$.
Notice that market efficiency is equivalent to rational expectations, one must text a model of
expected returns as well when testing market efficiency. After defining a model of expected returns,
the variables to be included in the information set must be specified. \citet{malkiel1970efficient} define
three forms of efficient market hypothesis and the corresponding information sets:
\begin{enumerate}
    \item[-] the \textbf{\textit{weak form}}: past returns
    \item[-] the \textbf{\textit{semi-strong form}}: publicly available information such as stock splits, dividends, or earnings
    \item[-] \sidenotes{$\leftarrow$ this could be tested by using measureable actions (trades or portfolio holdings) of the potentially better informed agents} the \textbf{\textit{strong form}}: information available to some market participants, but NOT necessarily to all participants.
\end{enumerate}

In the time-series literature, the simplest economic model is constant return: $\Theta_{i,t}=\Theta$. In Section \ref{chap1:sec2}, I summarize
the early literature focusing on this model.

Market efficiency has been widely tested and debated, now the most accepted view of market efficiency hypothesis is that
it is a useful benchmark but does not hold perfectly. The debates between long-term versus short-term efficiency, micro versus
macro efficiency are still and will continue to be heated. Some noticable alternative hypotheses are:
\begin{enumerate}
    \item[-] \textbf{\textit{High-frequency noise}}: market prices are contaminated by short-term noise, which can be caused by measurement errors or illiquidity (bid-ask bounce).
    \item[-] \textbf{\textit{Inperfect information processing}}: the market reacts sluggishly to information after its releasing
    \item[-] \textbf{\textit{Persistent mispricing}}: market prices deviate substantially from efficient levels in a LONG time
    \item[-] \textbf{\textit{Disposition effect}}: individual investors are more willing to sell winning stocks then losing stocks, see \citet{shefrin1985disposition} for details.
\end{enumerate}

\subsection{Model: autocorrelation of returns}
The most basic time-series test of market efficiency is to test "whether past deviations of returns from model-implied expected returns
predict future return deviations" \citep[See][p.~124]{campbell2017financial}. 