In this section, I follow \citet[Chapter~5]{campbell2017financial} and discuss some 
of the conceptual building blocks for the strand of time-series empirical finance literature.

\subsection{Market Efficiency}
An intuitive way of explaining \textbf{\textit{market efficiency}} is that efficient markets
are competitive and allow no easy ways to make economic profit. A more useful and testable
definition was given by \citet[p.~127]{malkiel1989efficient}:

\begin{quote}
    The market is said to be efficient with respect to some information set $\phi$,
    if security prices would be unaffected by revealing that information to all participants.
\end{quote}

Some event studies that measure market responses to news announcements can be interpreted
as tests of market efficiency regarding the announced information, but in general, this 
definition is not easy to test. On the other hand, \cite{malkiel1989efficient} gives a more testable alternative:
\begin{quote}
    Efficiency with respect to an information set $\phi$ implies that it is impossible to
    make economic profits by trading on the basis of $\phi$.
\end{quote}

This is the idea behind an enormous literature in empirical asset pricing: if an economic model
defines the equilibrium return as $\Theta_{i,t}$, then the null hypothesis is
\begin{equation}
    R_{i,t+1} = \Theta_{i,t}+U_{i,t+1}
\end{equation}
where $U_{i,t+1}$ is a FAIR game regarding the information set at t, or $\mathbb{E}(U_{i,t+1}|\phi_t)=0$.
Notice that market efficiency is equivalent to rational expectations, one must text a model of
expected returns as well when testing market efficiency.