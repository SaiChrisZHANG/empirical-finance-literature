With the models introduced in Section \ref{chap1:sec1:ssec2}, the mean reversion in stock returns
has been examined empirically over the years. The common ground is that there are autocorrelations,
but the directions of autocorrelations are indefinite across different settings.

A brief summary is listed below, and the details are discussed in the corresponding sub-sections.

\subsection{Mean Riverse: Negative autocorrelations}\label{chap1:sec2:ssec1}

\subsection{Positive autocorrelations}\label{chap1:sec2:ssec2}
Due to the positive cross-autocorrelations among individual stocks, \textbf{Stock indexes} have predominantly
positive high-frequency autocorrelations.

\subsubsection{\citet{lo1988stock,lo1990contrarian}}\label{chap1:sec2:ssec2:paper1}
\citet{lo1988stock,lo1990contrarian} presented evidence that broad market indexes have predominantly positive
high-frequency autocorrelations. In \citeyear{lo1988stock}, they found a 30\% AR(1) with weekly returns
of \textit{September 1962} to \textit{December 1985}, while higher-order ARs are also positive although smaller in magnitude.
In \citeyear{lo1990contrarian}, they found similar results with the sample period extended to \textit{December 1987}.
They proposed that the equal-weighted index autocorrelation could be rewritten into the sum of own-autocovariances
and cross-autocovariances of the component securities. Given that the autocorrelations of individual stock returns
are generally negative, they deducted that the cross-autocovariances must be positive and large enough to exceed the sum
of the negative own-autocovariances. They built the following model and showed the importance of the forecastability 
\textit{across} securities for contrarian profits:

Consider a stylized contrarian investment strategy: buy stocks at time $t$ that were losers at time $t-k$ and sell 
stocks at $t$ that were winners at $t-k$. This strategy can be formally written as 
$$
\omega_{i,t}(k)=-\frac{1}{N}(R_{i,t-k}-R_{m,t-k}),\ i=1,\cdots,N
$$
where $R_{m,t-k}=\frac{1}{N}\sum^N_{i=1}R_{i,t-k}$ is the market (equal-weighted) index return.

By construction, $\vec{\omega}_{t}(k)\equiv \left[\omega_{1,t}(k),\cdots,\omega_{N,t}(k)\right]'$ is an arbitrage portfolio 
since the weights sum to zero. Such a strategy is designed to take advantage of stock market overreactions since the stocks
whose returns deviate more from the market index return will be given higher weights (more positive for huge losers and 
vice versa). Profit generated from this strategy is $\pi_t(k)=\sum^N_{t=1}\omega_{i,t}(k)R_{i,t}$, rewrite this profit, get:

$$
\pi_t(k)= \sum^N_{i=1}\omega_{i,t}(k)R_{i,t}= -\frac{1}{N}\sum^N_{i=1}\left(R_{i,t-k}-R_{m,t-k}\right)R_{i,t} = -\frac{1}{N}\sum^N_{i=1}R_{i,t-k}R_{i,t}+R_{m,t-k}R_{m,t}
$$
take expectation, get:
\begin{align*}
    E[\pi_t(k)] &= -\frac{1}{N}\sum^N_{i=1} E\left[R_{i,t-k}R_{i,t}\right]+E\left[R_{m,t-k}R_{m,t}\right]\\
    &= -\frac{1}{N}\sum^N_{i=1}\left(Cov\left[R_{i,t-k},R_{i,t}\right] +\mu_i^2 \right) + \left(Cov\left[R_{m,t-k},R_{m,t}\right] +\mu_m^2 \right)\\
\end{align*}
where $\mu_m\equiv E\left[R_{m,t}\right]=\mu'\iota/N$. Reorganizing this equation into 3 components:
\begin{align*}
    E[\pi_t(k)] &= -\frac{1}{N}tr(\Gamma_k) -\frac{1}{N}\sum^N_{i=1}\mu_i^2+\frac{\iota'\Gamma_k\iota}{N^2}+\mu_m^2\\
    & = \underbrace{\left\{\frac{\iota'\Gamma_k\iota}{N^2} -\frac{1}{N} tr(\Gamma_k) \right\}}_{\mathbf{L_k}}- \left\{\frac{1}{N}\sum^N_{i=1}(\mu_i-\mu_m)^2\right\}\\
    & = \underbrace{\frac{1}{N^2}\left[\iota'\Gamma_k\iota-tr(\Gamma_k)\right]}_{\mathbf{C_k}} + \underbrace{\left(-\frac{N-1}{N}\right)tr(\Gamma_k)}_{\mathbf{O_k}} - \underbrace{\frac{1}{N}\sum^N_{i=1}(\mu_i-\mu_m)^2}_{\sigma^2(\mu)}
\end{align*}
where $tr(\cdot)$ indicates the trace operator (sum of diagonal elements).

The three components are:
\begin{enumerate}
    \item[-]: $\mathbf{C_k}$ depends ONLY on the off-diagonals of the auto-covariance matrix $\Gamma_k$
    \item[-]: $\mathbf{O_k}$ depends ONLY on the diagonals of the auto-covariance matrix $\Gamma_k$
    \item[-]: $\sigma^2(\mu)$ is independent of $\Gamma_k$
\end{enumerate}
Or, cross-autocovariances dictate $\mathbf{C_k}$, own-autocovariances dicate $\mathbf{O_k}$, this is the separation needed.

With this decomposition, the authors explained that the profitability of such a contrarian strategy could be perfectly
consistent with a \textbf{positively autocorrelated market index} and \textbf{negatively autocorrelated individual security returns}.

They also presented 5 illustrative scenarios:
\begin{enumerate}
    \item[A] \underline{\textbf{\textit{I.I.D. returns}}}
    \begin{enumerate}
        \item[-] $\forall k, \Gamma_k=0, L_k=C_k=O_k=0$, thus, $E[\pi_t(k)]=-\sigma^2(\mu)\leq 0$. 
        \item[-] \textit{\textbf{Intuition}}: When returns do follow random walks, any cross-sectional 
        variation in expected returns would generate negative expected profits when trading with a contrarian strategy. 
        Since the strategy is reduced to shorting the higher and buying the lower mean return securities. BUT, $\sigma^2(\mu)$ is generally small.
    \end{enumerate}

    \item[B] \underline{\textbf{\textit{Stock market overreaction}}}
    \begin{enumerate}
        \item[-] \textbf{\textit{Assumption}}: negative self-autocorrelations and zero cross-autocorrelations, i.e., 
        the diagonal elements of $\Gamma_k$ are negative, the non-diagonal elements are 0. Thus, ignoring the small $\sigma^2(\mu)$, 
        the expected profit is $$E[\pi_t(k)]\simeq L_k = O_k = -\left(\frac{N-1}{N}\right)tr(\Gamma_k) = -\left(\frac{N-1}{N}\right)\sum^N_{i=1}\gamma_{ii}(k)>0$$ 
        \item[-] \textit{\textbf{Intuition}}: In a overreacting market, "what goes up must come down" and vice versa. Thus, a contrarian investment strategy is profitable on average.
        A special case is the model of "fads": the sum of a random walk and an AR(1)\footnote{Notice that only AR(1) statisfies this conclusion.}.
    \end{enumerate}
    \item[C] \underline{\textbf{\textit{White noise and lead-lag relations}}}
    \begin{enumerate}
        \item[-] \textbf{\textit{Assumption}}: $\forall i \in [1,\cdots,N]$, the return is given by: $R_{i,t}=\mu_i+\beta_i\Lambda_{t-i}+\epsilon_{i,t}$, where $\beta_i>0$,
        $\Lambda$ is a serially independent common factor with zero mean and variance $\sigma_{\lambda}^2$, $\epsilon_{i,t}$ is assumed to be both cross-sectionally and serially independent.
        When $k<N$, the autocovariance matrix $\Gamma_k$ has zeros in all entries except along the $k$th superdiagonal. On the $k$th superdiagonal, 
        the elements are $\gamma_{i,i+k}=\sigma^2_{\lambda}\beta_i\beta_{i+k}$, the profit is: $$E[\pi_t(k)]\simeq L_k = C_k = \frac{\sigma^2_{\lambda}}{N^2}\sum^{N-k}_{i=1}\beta_i\beta_{i+k}>0$$
        \item[-] \textbf{\textit{Intuition}}: This is an artifact of the dependence of the $i$th security's return on a lagged common factor, where the lag is determined by $i$. Notice that the returns
        are serially independent, but sotck $i$'s returns can be predicted with past returns of stock $j$, where $j<i$. With this cross-correlation, a contrarian strategy can still profit, as long
        as the cross autocovariances are sufficiently large.
    \end{enumerate} 
    \item[D] \underline{\textbf{\textit{Nonsynchronous trading and lead-lag effects}}}\footnote{See \citet{lo1990econometric} for a detailed discussion.}
    \begin{enumerate}
        \item[-] \textbf{\textit{Assumption}}:
        
        \underline{\textit{"Virtual" return}}: the returns for security $i\in[1,\cdots,N]$ are generated by: $R_{i,t}=\mu_i+\beta_i\Lambda_i+\epsilon_{i,t}$. $\Lambda_t$ is
        some zero-mean, i.i.d. common factor, $\epsilon_{i,t}$ is zero-mean, serially and cross-sectionally independent. But this time the returns are \textit{unobservable}, i.e., "virtual".
        
        \underline{\textit{Non-trade}}: in each period $t$, security $i$ has an i.i.d. probability of $p_i$ to be NOT traded. If not traded, a security's \textit{observed} return $R^o_{i,t}$ is 0 while its
        true return is still given by $R_{i,t}=\mu_i+\beta_i\Lambda_i+\epsilon_{i,t}$. 
        
        \underline{\textit{Observed return}}: The observed return of security $i$ in period $t$ is the sum of its virtual returns of all the past \textbf{consecutive non-trading} periods.
        Formally, $R^o_{i,t}=\sum^{\infty}_{k=0}X_{i,t}(k)R_{i,t-k}$. The weights $X_{i,t}(k)\equiv(1-\delta_{i,t})\delta_{i,t-1} \cdots \delta_{i,t-k}$,
        where $\delta_i$ is the (i.i.d.) non-trading indicator. 
        $X_{i,t}(k)$ is also an indicator:
        \begin{equation*}
            X_{i,t}(k) =
              \begin{cases}
                1 & \text{$i$ is traded at $t$, but not in any of the $k$ previous periods}\\
                0 & \text{otherwise}
              \end{cases}       
          \end{equation*}
        
        \underline{\textit{Nontrading duration}}: the nontrading duration, or the number of past consecutive periods that security $i$ is not traded, 
        is $\tilde{k}_{i,t}\equiv \sum^{\infty}_{k=1}\left(\prod_{j=1}^k\delta_{i,t-j}\right)$, its expectation is $E[\tilde{k}_{i,t}]=\frac{p_i}{1-p_i}$.
        
        \underline{\textit{Portfolio return}}: for an equal-weighted portfolio of securities with common nontrading probability $p_{\kappa}$, the observed return to
        portfolio can be approximated as $R_{port,t}^o\rightarrow \mu_{port}+(1-p_{port})\beta_{port}\sum^{\infty}_{k=0}p_{port}^k\Lambda_{t-k}$, where $\beta_{port}$ is the
        average $\beta$ of the securities. Then the observed return of the portfolio over $q$ periods is $R^o_{port,T}(q)\equiv \sum^{Tq}_{t=(T-1)q+1}R^o_{port,t}$.
    
        \item[-] \textbf{\textit{Intuition}}:
        
        The "nontrading" problem aims to fix one problem: the prices of distinct securities are mistakenly assumed to be sampled simultaneously. Prices actually happen
        in different periods, but are treated as if they were observed at the same time. The "power" of a stock on others is related to how frequently it is traded: For
        a more frequently traded portfolio $a$, and a less frequently traded portfolio $b$, $R_{a,t-1}$ predicts $R_{b,t}$ better than $R_{b,t-1}$ predicts $R_{a,t}$.
        HOWEVER! This cannot fully explain the magnitude of weekly cross-autocorrelations.
    
    \end{enumerate} 
    \item[E] \underline{\textbf{\textit{Positively dependent common factor and bid-ask spread}}}
    \begin{enumerate}
        \item[-] $R_{i,t}$ as the sum of: (a) a positively autocorrelated common factor, (b) idiosyncratic white noise, (c) a bid-ask spread process. Formally, 
        $$R_{i,t}=\mu_i+\beta_i\Lambda_i+\eta_{i,t}+\epsilon_{i,t}$$
        where $E[\Lambda_t]=0$, $E[\Lambda_{t-k}\Lambda_t]\equiv\gamma_{\lambda}(k)>0$ (positively autocorrelated common factor), $E[\epsilon_{i,t}]=E[\eta_{i,t}]=0$ (idiosyncratic noise),
        $Var[\epsilon_{i,t}]=\sigma^2_i$.

        The bid-ask spread has a AR(1) as $E[\eta_{i,t-1}\eta_{i,t}]=-s_i^2/4$, where $s_i$ is the percentage bid-ask spread. All higher-order ARs and all cross-correlations are zero.

        The autocovariance matrices are given by:
        $$\Gamma_k =
        \begin{cases}
            \gamma_{\lambda}(1)\beta\beta' -\frac{1}{4}diag[s_1^2,\cdots,s_N^2],&k=1\\
            \gamma_{\lambda}(k)\beta\beta',&k>1
        \end{cases}
        $$
    \end{enumerate} 
\end{enumerate}

Based on the size-sorted portfolio returns, the authors have found that returns on larger and more liquid stocks 
and subsequent returns on smaller and less liquid stocks are positively cross-autocorrelated. 



