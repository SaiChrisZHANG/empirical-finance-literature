With the models introduced in Section \ref{chap1:sec1:ssec2}, the mean reversion in stock returns
has been examined empirically over the years. The common ground is that there are autocorrelations,
but the directions of autocorrelations are indefinite across different settings.

A brief summary is listed below, and the details are discussed in the corresponding sub-sections.

\subsection{Mean Riverse: Negative autocorrelations}\label{chap1:sec2:ssec1}

\subsection{Positive autocorrelations}\label{chap1:sec2:ssec2}
Due to the positive cross-autocorrelations among individual stocks, \textbf{Stock indexes} have predominantly
positive high-frequency autocorrelations.

\subsubsection{\citet{lo1988stock,lo1990contrarian}}\label{chap1:sec2:ssec2:paper1}
\citet{lo1988stock,lo1990contrarian} presented evidence that broad market indexes have predominantly positive
high-frequency autocorrelations. In \citeyear{lo1988stock}, they found a 30\% AR(1) with weekly returns
of \textit{September 1962} to \textit{December 1985}, while higher-order ARs are also positive although smaller in magnitude.
In \citeyear{lo1990contrarian}, they found similar results with the sample period extended to \textit{December 1987}.
They proposed that the equal-weighted index autocorrelation could be rewritten into the sum of own-autocovariances
and cross-autocovariances of the component securities. Given that the autocorrelations of individual stock returns
are generally negative, they deducted that the cross-autocovariances must be positive and large enough to exceed the sum
of the negative own-autocovariances. They built the following model and showed the importance of the forecastability 
\textit{across} securities for contrarian profits:

Consider a stylized contrarian investment strategy: buy stocks at time $t$ that were losers at time $t-k$ and sell 
stocks at $t$ that were winners at $t-k$. This strategy can be formally written as 
$$
\omega_{i,t}(k)=-\frac{1}{N}(R_{i,t-k}-R_{m,t-k}),\ i=1,\cdots,N
$$
where $R_{m,t-k}=\frac{1}{N}\sum^N_{i=1}R_{i,t-k}$ is the market (equal-weighted) index return.

By construction, $\vec{\omega}_{t}(k)\equiv \left[\omega_{1,t}(k),\cdots,\omega_{N,t}(k)\right]'$ is an arbitrage portfolio 
since the weights sum to zero. Such a strategy is designed to take advantage of stock market overreactions since the stocks
whose returns deviate more from the market index return will be given higher weights (more positive for huge losers and 
vice versa). Profit generated from this strategy is $\pi_t(k)=\sum^N_{t=1}\omega_{i,t}(k)R_{i,t}$, rewrite this profit, get:

$$
\pi_t(k)= \sum^N_{i=1}\omega_{i,t}(k)R_{i,t}= -\frac{1}{N}\sum^N_{i=1}\left(R_{i,t-k}-R_{m,t-k}\right)R_{i,t} = -\frac{1}{N}\sum^N_{i=1}R_{i,t-k}R_{i,t}+R_{m,t-k}R_{m,t}
$$
take expectation, get:
\begin{align*}
    E[\pi_t(k)] &= -\frac{1}{N}\sum^N_{i=1} E\left[R_{i,t-k}R_{i,t}\right]+E\left[R_{m,t-k}R_{m,t}\right]\\
    &= -\frac{1}{N}\sum^N_{i=1}\left(Cov\left[R_{i,t-k},R_{i,t}\right] +\mu_i^2 \right) + \left(Cov\left[R_{m,t-k},R_{m,t}\right] +\mu_m^2 \right)
\end{align*}

Based on the size-sorted portfolio returns, the authors have found that returns on larger and more liquid stocks 
and subsequent returns on smaller and less liquid stocks are positively cross-autocorrelated. 



