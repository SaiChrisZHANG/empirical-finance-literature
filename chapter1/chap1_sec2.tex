With the models introduced in Section \ref{chap1:sec1:ssec2}, the mean reversion in stock returns
has been examined empirically over the years. The common ground is that there are autocorrelations,
but the directions of autocorrelations are indefinite across different settings.

A brief summary is listed below, and the details are discussed in the corresponding sub-sections.

\subsection{Positive autocorrelations}
Due to the positive cross-autocorrelations among individual stocks, \textbf{Stock indexes} have predominantly
positive high-frequency autocorrelations.

\citet{lo1988stock,lo1990contrarian} found that returns on larger and more liquid stocks and subsequent
returns on smaller and less liquid stocks are positively cross-autocorrelated. This is the driving force of 
the positive autocorrelation of market index. In \citeyear{lo1988stock}, they found a 30\% AR(1) with weekly returns
of July 1962 to December 1987, while higher-order ARs
are also positive although smaller in magnitude.


\subsection{Negative autocorrelations}

